\section{Introduction}
Artificial Intelligence is reshaping America’s \$9.4 trillion labor market, creating cascading effects across industries and communities that extend beyond visible technology sectors. When AI systems automate quality control in automotive manufacturing, the consequences propagate through supplier networks, logistics operations, and local service economies. These ripple effects multiply the economic stakes of AI adoption, yet remain invisible to the systems that guide billion-dollar workforce and infrastructure decisions.

While consumer applications such as ChatGPT capture public attention - helping with homework, writing, or quick research - the larger restructuring is already underway inside firms. AI systems now generate more than a billion lines of code each day, prompting companies to restructure hiring pipelines and reduce demand for entry-level programmers~\cite{TrueUp2025}. These observable changes in technology occupations signal a broader reorganization of tasks that extends beyond software development.

AI capabilities now span diverse operational contexts. Financial institutions deploy AI for document processing and analytical support. Healthcare systems automate administrative tasks, enabling clinical staff to allocate more time to patient care. Logistics operations integrate AI-powered systems to optimize fulfillment while creating demand for maintenance and coordination roles. Manufacturing facilities deploy AI-driven quality control to automate inspection tasks while creating roles in system coordination and oversight. Analysis of Bureau of Labor Statistics skill taxonomies reveals that current AI systems can technically perform approximately 16 percent of classified labor tasks (see Appendix A) - yet this technical capability remains largely invisible to the workforce planning systems guiding billion-dollar investments.
% These developments illustrate how AI adoption is expanding  reshaping work faster than traditional metrics can track

Historical precedent demonstrates both the opportunities and risks of such technological shifts. The internet revolution generated substantial economic value: technology companies now represent over one-third of S\&P 500 market capitalization~\cite{ssga_spy_2025} and the U.S. digital economy contributes approximately \$4.9 trillion (\~18\%) to GDP~\cite{bea_2023_infographic}. The impact extends beyond technology firms - research indicates each high-technology job correlates with approximately five additional positions in local services, healthcare, and retail ~\cite{moretti2012,brookings2012}. Early-moving regions captured durable advantages during the internet era: North Carolina’s Research Triangle matured into a global research hub~\cite{rtp_case_2018}; Texas scaled Austin into a top tech market~\cite{cbre_2024_austin}; Tennessee and Kentucky became national logistics leaders as FedEx’s Memphis super-hub and UPS Worldport in Louisville expanded~\cite{memphis_airport_cargo,ups_worldport_2022}; and Utah’s “Silicon Slopes” rose as a cloud computing center~\cite{cbre_2024_slc}.

Similar dynamics will now shape AI adoption patterns. Regions that align skill development, infrastructure investment, and industry strategy early may establish competitive advantages, while delayed response could result in widening disparities. Project Iceberg provides analytical infrastructure to support evidence-based planning as AI capabilities expand across the economy.

% States that moved early captured durable advantage: North Carolina’s Research Triangle matured into a global research hub~\cite{rtp_case_2018}; Texas scaled Austin into a top tech market~\cite{cbre_2024_austin}; Tennessee and Kentucky became national logistics leaders as FedEx’s Memphis super-hub and UPS Worldport in Louisville expanded~\cite{memphis_airport_cargo,ups_worldport_2022}; and Utah’s “Silicon Slopes” rose as a cloud/SaaS center~\cite{cbre_2024_slc}. The same dynamic may now shape AI: regions that align education, infrastructure, and industry early will thrive, while late movers risk falling behind. Project Iceberg provides the workforce intelligence to ensure AI strengthens - rather than fragments - communities.

% Directly, the U.S. digital economy contributes about \$2.6 trillion to GDP~\cite{bea_2023_infographic}. Yet the impact extends well beyond tech firms: each high-tech job is associated with roughly five additional local jobs in services, healthcare, and retail~\cite{moretti2012,brookings2012}.


\begin{figure}[t!]
    \centering
    \includegraphics[width=0.97\linewidth]{images/iceberg_why.png}
    \caption{Traditional workforce metrics miss AI-mediated tasks. Census data captures jobs tied to geographic locations and business addresses. Human-AI collaboration—where workers and AI systems jointly perform tasks within occupations—creates new forms of labor that existing metrics don't capture. The Iceberg Index provides a forward-looking measure of this technical automation exposure, revealing skill-based transformation that remains invisible.}
    \label{fig:placeholder}
\end{figure}

\section{Measurement Gaps in Workforce Planning}
Federal and state governments are already committing billions to prepare for the AI economy. The federal AI Action Plan~\cite{whitehouse2025ai} outlines 90 policy positions and has launched a national AI Workforce Research Hub~\cite{whitehouse2025ai}. DOE has identified 16 federal sites for data center development~\cite{doe2025sites}. 

States are responding with major initiatives: 
\begin{itemize}
    \item North Carolina has secured a \$10 billion investment to expand data center capacity~\cite{amazon2025nc}
    \item Tennessee's Google-Kairos reactor will power data centers while anchoring Oak Ridge as a nuclear innovation hub~\cite{google2025tennessee}
    \item Utah's Operation Gigawatt will double statewide clean-energy production over the next decade~\cite{utah_operation_gigawatt_2025}
    \item Virginia has committed \$1.1 billion to train 32,000 AI graduates~\cite{vedp2025datacenters}
    \item DOE has also committed \$100 million for nuclear safety training as that workforce is projected to triple by 2050~\cite{doe2025nuclear}
\end{itemize}
These efforts reflect real momentum in infrastructure, energy, and talent development.

\noindent Yet evidence suggests workforce change is occurring faster than planning cycles can accommodate. Payroll data covering millions of workers shows a 13\% relative decline in early-career employment (ages 22–25) for AI-exposed occupations~\cite{stanford_workforce}. Analysis of job postings across 285,000 firms for 62 million workers reinforces the pattern: the demand for entry-level positions
have subdued while the focus has shifted to hiring for experienced roles~\cite{harvard_workforce}. These shifts, alongside widespread restructuring in the technology sector~\cite{TrueUp2025}, indicate that the pace of change is accelerating across the economy. States are committing billions to workforce programs while key workforce dynamics remain invisible to traditional planning tools.

\subsection{Challenge 1: Anticipating Workforce Shifts Before They Happen}
The labor market is evolving faster than current data systems can capture. AI automates some skills and augments others, creating uneven effects across industries and regions. Much of this activity occurs on digital platforms—gig marketplaces, AI copilots, freelance networks—that fall outside conventional reporting~\cite{oecd_platform_gap_2019}. By the time these changes appear in official statistics, policymakers may already be reacting to yesterday’s disruptions, committing billions to programs that target skills already displaced. Without forward-looking capability to test strategies before implementation, states cannot distinguish investments that prepare workers from those that arrive too late.

% Official statistics capture these changes with significant delay. By the time workforce disruption appears in employment data, it reflects conditions from months or quarters earlier. When policymakers design programs based on this delayed information, they risk investing billions in solutions for skill gaps that have already shifted or evolved. Without tools to model how AI capabilities might spread across occupations under different scenarios, leaders cannot test whether their strategies address emerging needs or yesterday's problems.

% These invisible shifts interact with feedback loops: as technology matures, firms adopt new tools, and states respond with training and investment. But by the time these changes appear in official statistics, policymakers may already be reacting to yesterday’s disruptions, committing billions to programs that arrive too late or target skills already displaced. Without forward-looking simulations to anticipate these dynamics, leaders lack the ability to test how strategies will perform as AI capabilities spread across occupations.

\begin{tcolorbox}[colback=white,colframe=IcebergDark]
\centering \textbf{The Invisible Economy:} \\
AI adoption and worker behavior unfold on platforms outside official data. Feedback loops between technology maturity, firm adoption, and state interventions mean that by the time disruptions appear in surveys, policy responses may target skills already displaced or industries already transformed.
\end{tcolorbox}

\subsection{Challenge 2: Measuring Human–AI Work, Not Just Human Work.}
Existing workforce planning frameworks were designed for human-only economies. They track employment, wages, and productivity, but were not designed to measure where AI capabilities overlap with human skills before adoption reshapes occupational structure. Each major transition required new measurement: in the postwar era, output per hour captured physical productivity~\cite{bls_productivity_intro}; in the internet era, the Digital Economy Satellite Account measured the value of online services~\cite{bea_desa_2023}. The AI era is defined by intelligence itself becoming a shared input between humans and machines. What matters is not only the number and complexity skills to execute, but how much of each skill's value can be performed by AI and which human skills remain differentially valuable. Without a skills-centered metric, states lack a systematic way to align investments with where AI-human skill overlap is concentrated.

\begin{tcolorbox}[colback=white,colframe=IcebergDark]
\centering \textbf{Every Revolution Needs a New Metric:} \\
Industrial era $\rightarrow$ Output per hour (measured physical productivity) \\
Internet era $\rightarrow$ Digital economy accounts (captured online service value) \\
Intelligence era $\rightarrow$ Skills-centered measure (reveals AI-human skill overlap)
\end{tcolorbox}

% that early-career workers (ages 22-25) in AI-exposed occupations have experienced a 13\% relative decline in employment since widespread AI adoption began~\cite{stanford_workforce}. Analysis of job posting data for 62 million workers across 285,000 firms reinforces these observations, finding that companies adopting AI continue hiring experienced workers while reducing entry-level hiring~\cite{harvard_workforce}. These patterns occur alongside broader workforce restructuring, especially in the technology sector~\cite{TrueUp2025}.

% While this infrastructure buildout accelerates, states have launched major workforce preparation initiatives. Virginia launched a \$1.1 billion Tech Talent Investment Program to train 32,000 AI graduates~\cite{vedp2025datacenters} and DOE committed \$100 million for nuclear safety training as the sector workforce is projected to triple by 2050~\cite{doe2025nuclear}. These investments demonstrate the awareness for state leadership in preparing the workforce for the AI economy.

% However, workforce transformation is already happening faster than these investment timelines anticipated. Recent analysis of payroll data covering millions of U.S. workers shows that early-career workers (ages 22-25) in AI-exposed occupations have experienced a 13\% relative decline in employment since widespread AI adoption began~\cite{stanford_workforce}. Analysis of job posting data for 62 million workers across 285,000 firms reinforces these observations, finding that companies adopting AI continue hiring experienced workers while reducing entry-level hiring~\cite{harvard_workforce}. These patterns occur alongside broader workforce restructuring, especially in the technology sector~\cite{TrueUp2025}.

% These converging signals indicate that workforce transformation is accelerating across the economy, yet states lack the capability to test how their billion-dollar workforce programs will perform against these emerging patterns.

% \textbf{\textcolor{IcebergDark}{Challenge 1: Anticipating Workforce Shifts Before They Happen}}: 
% \\
% The labor market is evolving faster than current data systems can capture. AI automates some skills and augments others, creating uneven effects across industries and regions. Much of this activity happens on digital platforms—gig platforms, AI copilots, freelance marketplaces—that traditional surveys and administrative data miss~\cite{oecd_platform_gap_2019}. Without forward-looking simulations to see through this “invisible economy”, leaders risk mis-allocating billions in training and infrastructure.

% \textbf{\textcolor{IcebergDark}{Challenge 2: Planning Human–AI Work, Not Just Human Work}}:
% \\
% Current workforce planning frameworks were designed for earlier economies. They track employment levels, wage trends, and productivity metrics, but cannot evaluate how human–AI collaboration will reshape state economies. History shows that each major transition required new measures: in the postwar era, physical productivity was captured through output-per-hour statistics~\cite{bls_productivity_intro}, and in the digital era, the Bureau of Economic Analysis created the Digital Economy Satellite Account to measure value added from digital services~\cite{bea_desa_2023}. The AI transition is no different: states now need planning infrastructure and new metrics that reveal where workforce dollars will have the greatest impact.

% Current workforce planning frameworks were designed for earlier economies. They track employment levels, wage trends, and productivity metrics, but cannot evaluate how human–AI collaboration will reshape state economies. History shows that every major transition required new measures: in the postwar industrial era, physical productivity was captured through systematic output-per-hour statistics~\cite{bls_productivity_intro}, and in the digital era, value added from software, cloud services, and e-commerce was captured through the Bureau of Economic Analysis’s Digital Economy Satellite Account~\cite{bea_desa_2023}. The AI transition is no different: states now need planning infrastructure and new metrics that reveal where workforce dollars will have the greatest impact.

% History shows that every major transition required new measures: physical output per hour captured through systematic labor statistics~\cite{bls_productivity_intro}, and value added from digital services were captured through the Bureau of Economic Analysis’s Digital Economy Satellite Account~\cite{bea_desa_2023}. The AI transition is no different: states now need planning infrastructure and new metrics that reveal where workforce dollars will have the greatest impact.

% \textbf{\textcolor{IcebergDark}{Challenge 2: Planning Human–AI Work, Not Just Human Work}}: 

% Current workforce planning frameworks were designed for human-only economies. They track employment levels, wage trends, and productivity metrics, but cannot evaluate how human-AI collaboration will actually impact state economies. As AI changes how work gets done, traditional budget indicators provide incomplete guidance for workforce investment decisions. States need strategic planning infrastructure that show where to allocate workforce dollars for maximum impact in the new human-AI economy.


